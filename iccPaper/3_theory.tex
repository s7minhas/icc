\section*{Theoretical Framework}

Existing research suggests that international organizations have some degree of autonomy from the states that created them.  IOs derive authority and independence through the delegation processes that state engage in when creating them, and through the expertise and specialization that they provide (Barnett and Finnemore 2004, 43).  

The ICC, specifically, is imbued with rational-legal authority as a result of the Rome Statue ratification process which brought it into being.  

Given that the ICC's situation-selection and advancement process is largely autonomous from the states that created it, it is imperative to understand the incentives and constraints of decision-makers at the court in order to build a coherent theory of ICC behavior.

In creating the ICC, states delegated to the Court the authority and autonomy to select situations for examination and investigation (Danner 2003; Schabas 2011; Stahn 2009).  While decision-making at the ICC is legally autonomous from member states and other international actors, that does not mean the court is indifferent to or unaffected by states.  Rather, we assume that the ICC and in particular the OTP strategically selects situations to examine and potentially formally investigate.\footnote{Arguably, there are greater restrictions, or greater state input on the first of these decisions.  We discuss this further… FIX THIS FOOTNOTE, this is a really important footnote to justify why we think the ICC has discretion over the first, but also to acknowledge they have more control over the second.} This is because the Court, despite its independence, must still be perceived as legitimate by members of the international community to maintain its institutional viability, or even survival.\footnote{This assumption is in line with sociological-institutionalist approaches, which argue that ``organizational survival and acceptance are dependent on demonstrating legitimacy'' (Barnett and Finnemore 2004, 43) (ADD SUCHMAN 1995 CITATION TOO).}  

Specifically, legitimacy is critical for the ICC for at least two reasons; first, the OTP and ICC require the cooperation of governments and other actors to properly investigate potential perpetrators. For example (Arrests, access, etc). Second, states that view the ICC as illegitimate may withdraw from it. Indeed, we observe this very problem occurring today, as several African states and even the African Union (AU) have either threatened to withdraw from the Court or even left it over its perceived bias against African states.  As a result of these consequences, we argue that the court selects situations that it expects will maintain or increase its legitimacy in the eyes of the international community, broadly speaking, and member states more specifically. 

Importantly, the Court can derive legitimacy through multiple different pathways.\footnote{For example, research suggests that IOs benefit from procedural legitimacy, or whether their ``procedures are viewed as proper and correct'', as well as substantive legitimacy, or whether they are ``reasonably successful at pursuing goals that are consistent with the values of the broader community'' (Barnett and Finnemore 2004, 166).} In the sections below, we develop two different strategic logics of ICC decision-making.  The first suggests that the ICC selects situations for examination and investigation most in line with its institutional mandate (i.e. to try the gravest cases that domestic courts cannot or will not take on).  As we argue below, acting in line with the court's legal mandate is important to the ICC's legitimacy because “states have little incentive to maintain support for an IO that routinely acts contrary to the mission that member states have assigned to it” (Beardsley and Schmidt 2012, 39).  The second argument below suggests that ICC decision-making is affected by the need to maintain the support of strong states whose support allows the court to function effectively.  This selection strategy has implications for the court's legitimacy because IOs are often ``judged by what they accomplish, and \ldots lack of effectiveness injures their legitimacy'' (Barnett and Finnemore 2004, 168).  

Importantly, these two selection strategies are neither mutually exclusive, nor always fully compatible with one another.  The ICC, therefore, faces a tradeoff; pursuing cases in line with powerful states' interests may improve legitimacy by making it more likely that the court gains the backing of key international actors to support its investigations and prosecutions, but doing so may also undermine legitimacy by deepening the belief that the court is a pawn of powerful states.  On the other hand, selecting situations purely based upon issues of gravity and complementarity may improve legitimacy by demonstrating the court's strict adherence to the mandate given to it by member states, but may undermine the court's ability to effectively carry out its examinations and investigations if the gravest cases are threatening to powerful states' interests and thus are not supported by key international actors.

\subsection*{Case Selection based on Legal Mandate}

As noted above, one central way international organizations maintain legitimacy is by acting in accordance with their institutional mandates (Barnett and Finnemore 2004; Beardsley and Schmidt 2012; Finnemore 2009).\footnote{This is, furthermore, consistent with a neoliberal or rational-functionalist accounting of IO behavior, which stresses that ``states have little incentive to maintain support for an IO that routinely acts contrary to the mission that member states have assigned to it'' (Beardsley and Schmidt 2012, 39).} Continual discrepancies between the goals and norms established in the Rome Statute and the actual behavior of ICC prosecutors would undermine the Court's legitimacy, and therefore the support it receives from state parties and the international community more generally.  Therefore, the court has incentives to select situations for examination and investigation that demonstrate its dedication to the mandate established in the Rome Statue.

We identify two key aspects of the ICC's institutional mandate that should drive the court's decision-making.  First, the Rome Statute tasks the ICC with investigating and prosecuting individuals who have committed the ``most serious crimes of concern to the international community as a whole'' (Rome Statue 5(1)).  That is, the court is expected to focus its attentions on the world's gravest crimes.  In fact, this criterion is written into the Rome Statute's sections on admissibility: Article 17(1)(d) of the Statute allows the Court to declare a case inadmissible ``when it is not of sufficient gravity to justify further action by the Court'' (Schabas 2011, 200). Therefore, the ICC's central mandate is to pursue justice for the world's worst abuses of human rights.\footnote{Focusing on the gravest abuses of human rights, furthermore, should help the court achieve one of its secondary goals – the deterrence of future human rights abuses.  Deterrence was a central concern of the crafters of the ICC, and an issue that the Court has taken up directly by noting that targeting the highest level perpetrators of the gravest human rights abuses is the best way to maximize the court's deterrent effects (Schabas 2011, 201, quoting from Lubanga arrest warrant decision). This relates closely to the idea of legitimacy: the ICC must build and maintain legitimacy in order to deter human rights violations, and it can only do so, or can do so most effectively, if it prosecutes the world's worst abusers of human rights.}   

The second aspect of the legal mandate that is central to case selection decisions centers on one of the Court's central admissibility principles. The ICC was conceived of and operates as a ‘court of last resort', meaning its jurisdiction extends only to situations in which domestic legal systems are unwilling or unable to effectively investigate or prosecute suspected violators of the law.  That is, ``The Court is intended to complement, not to replace, national systems. It can only prosecute if national systems do not carry out proceedings or when they claim to do so but in reality are unwilling or unable to carry out such proceedings genuinely'' (Bensouda 2012).  This is referred to as the complementarity principle, and is addressed in Article 17 of the Rome Statute, as well as paragraph 10 of the Preamble (Schabas 2011).  Evidence suggests that the OTP adheres to this complementarity principle when determining whether it has jurisdiction over particular situations: in all situations brought before the Court, the OTP has assessed the presence and legitimacy of national proceedings before moving forward with ICC proceedings (Bensouda 2012).\footnote{Currently, eight situations – those focused on Afghanistan, Colombia, Guinea, Iraq, Nigeria, Burundi, Israel, and Ukraine – are in this admissibility phase, where the Court examines and monitors domestic proceedings in order to determine whether further ICC action is warranted.}   

These two principles suggest that the OTP will initiate preliminary examinations and escalate its involvement to a formal investigation in situations characterized by the gravest human rights violations, and when domestic courts are unwilling or unable to hold perpetrators accountable.  We expect the gravity, or severity, of the abuses perpetrated by a government or non-state actors to be a strong, positive predictor of ICC initiation of a preliminary examination and escalation to a formal investigation.  The court's involvement will be directed, furthermore, at the actor specifically responsible for human rights abuses.  That is, the OTP will initiate examinations and advance formal investigations against agents and supporters of the state when the government has perpetrated grave abuses of human rights, and will initiate and escalate investigations against non-state actors (i.e. rebels and members of the opposition) when such non-state actors are responsible for grave human rights violations. 

Our second mandate-based expectation, based upon the complementarity principle, is that the court evaluates the likelihood of effective domestic prosecution before acting itself, pursuing investigations and prosecutions only in situations where domestic institutions fail to act in an unbiased, impartial manner. The likelihood of unbiased domestic prosecution, we argue, is a function of the effectiveness or independence of the state's judicial system (Conrad and Ritter 2013; Conrad 2014; Conrad and Moore 2010; Powell and Staton 2009). States with independent judicial institutions can more easily investigate and try perpetrators of gross human rights violations, even if those perpetrators are high-ranking government officials. In states without judicial independence, on the other hand, perpetrators can commit violations with domestic impunity. The state's judicial system can be manipulated such that government officials and citizens more generally can avoid prosecution should they commit acts proscribed by the ICC.  

However, assessing the presence and quality of domestic prosecutions is often undertaken as part of an ICC preliminary examination, in what the OTP labels Phase 3, or assessing admissibility.  We therefore expect that an independent judiciary will primarily impact OTP decisions on whether or not to advance from the preliminary examination to the formal investigation stage.  That is, complementarity is primarily a factor affecting the escalation of ICC involvement.  Our expectation, therefore, is that higher levels of judicial independence will be associated with fewer formal investigations, regardless of the target (government or opposition). 

\subsection*{Case Selection based on Power Politics}

Strict adherence to the mandate established in the Rome Statute is not the only factor affecting the ICC's legitimacy.  Research on institutional legitimacy suggests that an IO's perceived legitimacy is also affected by what it accomplishes, or its effectiveness (Barnett and Finnemore 2004; Suchman 1995, 580).  An institution that continually fails to accomplish its designated objectives will be perceived by member states as ineffective and will, as a result, suffer legitimacy costs that might ultimately threaten its viability.  (POSSIBLY ADD SENTENCE REFERENCING CURRENT CHALLENGES TO THE COURT ON EFFEFFECTIVENESS GROUNDS).  

An important determinant of IO effectiveness is whether it receives financial and political support from states, especially powerful ones (Barnett and Finnemore 2004, 168).  This is true for the ICC, as the court's institutional capacity, as established in the Rome Statute, is limited in several ways.  First, the ICC lacks enforcement capabilities. It has no standing army or police force and cannot execute its own warrants, and therefore relies upon the cooperation of states for enforcement (Goldsmith and Krasner 2003; Prorok 2017). The inability to enforce warrants/summonses on its own is a crucial limitation for at least two reasons. First, governments may lack the willingness or capability to capture and transfer wanted individuals themselves.\footnote{While some indicted individuals voluntarily turn themselves in, many suspects remain at large and must be arrested by states and transferred to the Court to face trial.}  Uganda, for instance, has been unable to capture wanted LRA leader Joseph Kony, and Ugandan president Museveni has requested the assistance of the United States to track him down. A government may also lack the willingness to turn over suspects, even if it has the capacity to do so. For example, the Ivory Coast government has refused to turn over Simone Gbago, wife of former President Laruent Gbago, to the ICC. Given that involved governments may often be unable or unwilling to enforce ICC warrants, pressure from third party states can play a crucial role in achieving compliance, either by aiding the unable or coercing the unwilling. The Court therefore relies upon support from third party states, especially powerful ones, in order to successfully prosecute wanted individuals.\footnote{\url{http://www.bbc.com/news/world-africa-24179992}}  

Further, it relies upon the cooperation and support of states to carry out its investigations. In the preliminary examination stage, the Court spends a considerable amount of time in a state to determine if it should move to a formal investigation. Once a formal investigation is underway, the Court must collect the necessary evidence to prosecute potential perpetrators. In the Democratic Republic of Congo, for instance, the OTP conducted more than 70 missions inside and outside of the DRC, while in Uganda it completed nearly 50 missions in a 10 month period (Human Rights Watch 2008). During its visits to Uganda, the OTP relied on the Ugandan Armed Forces as escorts to travel throughout the country due the instability there (Human Rights Watch 2008). Third party support may be necessary in many cases to persuade reluctant governments to cooperate with ICC investigators, or to financially support these fact-finding missions. Lacking state support, therefore, the Court could not operate effectively. 

How do these constraints affect the ICC's decision-making process? They are potentially quite important, as they may alter the Court's incentives. Ultimately, strong enough constraints may affect which situations the Court decides to examine or investigate.  That is, these constraints may undermine the Court's ability or willingness to efficiently fulfill its organizational mandate by pursuing justice in the gravest cases. The logic is as follows. As established above, the OTP is, in many cases, dependent upon powerful states for enforcement. In addition, the Court has an incentive to maximize its appearance of effectiveness and/or power. That is, the Court has incentives to limit actions that it believes will be directly refused or ignored, as this could weaken the institution, harming its reputation as a powerful institution and undermining its legitimacy.  Because of its dependence upon strong states and because its legitimacy and continued functioning depend upon its effectiveness as an institution, the OTP is unlikely to select cases for examination or investigation that threaten strong states' interests. Were the Court to lose the support of key powerful states upon which it relies for enforcement, the ICC's effectiveness at fulfilling its mandate would decrease, and its legitimacy would suffer.  Thus, the Court has incentives to maintain the support of the system's strongest states, in particular the five permanent members of the UN Security Council (P5). P5 support, or at least acquiescence, is key to the Court, despite the fact that three of the P5 – the United States, Russia, and China – are not state parties to the Rome Statute.  

First, P5 support is critical because these are the states with the power, resources, and international reach to act as the ICC's enforcers or, alternatively, to directly undermine the Court's effectiveness.  The United States, for example, despite its refusal to ratify the Rome Statute, has turned over suspects to the Court. In April 2013, US officials transferred fugitive warlord Bosco Ntaganda to the Hague after he turned himself in to a US Embassy in Rwanda (Simons 2013).  Similarly, as noted above, the US has worked with Uganda to capture LRA leader Joseph Kony. Second, because the Rome Statute gives the P5 the authority to refer cases to the ICC, the P5 constitute an important audience for the Court. The P5 came together in 2011 to refer the situation in Libya to the Court, demonstrating implicit support for the institution and bolstering its legitimacy. Russia and China, on the other hand, vetoed a UNSC resolution in May 2014 that would have referred the situation in Syria to the Court (``Syria War Crimes Move Blocked at UN'' 2014). 

Finally, the P5 are states with the coercive capacity to indirectly support or undermine the court. That is, they have a variety of carrots and sticks at their disposal that they can use to persuade third party states and other non-state actors to either support or undermine the ICC. The American Rewards for Justice Program, for example, bolsters the Court indirectly. While US law prohibits direct payments to the Court, this program supports the Court's activities by providing payments of up to five million dollars to third parties for information that leads to the apprehension of fugitives in atrocities cases (Simons 2013).  Likewise, several European states issued travel bans against indicted Kenyan leaders William Ruto and Uhuru Kenyatta, while President Obama refused to visit Kenya during his 2013 trip to Africa due to the ICC's indictments there.\footnote{\url{http://www.the-star.co.ke/news/article-99118/uhuru-ruto-banned-visiting-europe}} P5 actions are not always benevolent, however; reports indicate that Western diplomats exerted significant pressure on the Court not to open a Gaza war crimes inquiry (Borger 2014; ``Report: US Exerting Pressure on ICC Not to Open War Crimes Probe against Israel'' 2014). 

Ultimately, this suggests that P5 interests are likely to affect the Court's case selection decisions. Because the Court (1) relies upon strong states, particularly the P5, for enforcement support, (2) can have referrals blocked by the P5, and (3) receives significant indirect support and pressure from P5 states, the OTP is likely to take P5 interests into consideration as it determines which situations to pursue. The Court is, therefore, likely to pursue cases when it anticipates P5 support, whereas it may shy away from initiating preliminary examinations or escalating its involvement when P5 interests are mixed or are overtly threatened by such action. That is, the goal of pursuing justice in the worst situations where it cannot be served domestically may be undermined by the constraints the Court faces due to its reliance upon strong state support for enforcement, and its need to maintain support and effectiveness so as not to undermine its own legitimacy.

This logic suggests that the OTP will be less likely to initiate preliminary examinations and escalate its involvement in situations when such actions threaten P5 interests. This effect will be directed, furthermore: the OTP is less likely to initiate and escalate against agents and supporters of the state when the P5 have close ties to that state, and will be less likely to initiate and escalate investigations against non-state actors (i.e. rebels and members of the opposition) when P5 states have close ties to the relevant non-state actors. 

It is important to note that choosing to pursue cases in line with its legal mandate or with powerful states' interests can, in theory, bolster the ICC's legitimacy, these two sources of legitimacy are, at times, contradictory.  Pursuing one strategy, furthermore, can backfire, undermining the court's legitimacy rather than increasing it.  The functional need to prove efficacy, for example, requires the court to tailor its behavior to the specific interests of the powerful, but doing so may backfire, compromising the impartiality and nondiscriminatory prinicples of the institution.  As Barnett and Finnemore (2004, 169) note, IOs face a dilemma: ``effective performance requires reliance on powerful member states, but that same reliance can undercut an IO's substantive and procedural legitimacy when it undermines the appearance of impartiality and objectivity.'' 
