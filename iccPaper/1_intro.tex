On September 10, 2018, US National Security Advisor John Bolton issued a virulent critique of the International Criminal Court (ICC). Speaking to the Federalist Society, Bolton called the ICC ``ineffective, unaccountable, and indeed, outright dangerous'', and referred to the court as ``fundamentally illegitimate''.\footnote{The full text of Bolton's speech is available at: \citep{bolton_speech}.} While these statements reflect the policy positions of a US administration that is perhaps uniquely hostile to international institutions and efforts to advance international law, the critiques raised by Bolton are not new.

In fact, they mirror similar criticisms raised by a variety of states and international actors against the ICC over the past decade. In particular, criticism of the court has centered on two central issues: first, that the court is biased against African nations and other countries in the global south. This so-called ``Africa bias'' has prompted criticism from regional leaders such as Rwanda's Paul Kagame and key institutions, especially the African Union (\citep{bbc_news_2012, murithi2012international}. In fact, in 2016 Burundi, South Africa, and the Gambia threatened and/or initiated proceedings to leave the ICC based upon claims of bias \citep{murphy_2016}, and in 2018 the Philippines followed suit \citep{villamor2018philippines}. The second central critique of the court, mirrored in Bolton's speech, is that it is ineffective, incapable of fulfilling its mandate to achieve justice and deter atrocities. This criticism has gained traction in recent years as arrest warrants against key heads of state such as Omar Al-Bashir have remained unexecuted and the number of convictions remains at just four \citep{wang_2018, White_2018}.

These challenges to the ICC's legitimacy and effectiveness raise fundamental questions about its ability to fulfill its core mission to pursue international peace and justice. Proponents of the Court argue that it is a transformational institution with unprecedented authority to investigate and prosecute perpetrators of atrocity crimes, while critics argue the Court is biased and ineffective. Recent criticism makes clear that we must first understand where the court gets involved to fully understand its effects. Yet existing literature provides little rigorous empirical evidence regarding which of these views is more accurate. We are thus unable to answer the most basic question about the functioning of the Court: how does the ICC choose where it will act?

This paper addresses this gap by examining how the ICC's Office of the Prosecutor (OTP) selects situations for preliminary examination and formal investigation.\footnote{We use the term ``preliminary examination'' as shorthand to refer to both preliminary examinations, the first stage of proceedings when the OTP is acting pursuant to her Article 15 authority, and the ``pre-investigation phase'', or the first stage of proceedings when a situation is referred by the UNSC or a state party \citep{schabas2011introduction}. As we discuss further below, the OTP has considerable discretion when it comes to determining which situations and cases it will pursue \citep{danner2003enhancing, schabas2011introduction, stahn2009judicial}. We therefore focus our theoretical argument on the OTP chief prosecutor as the key actor of interest.}  We argue that the OTP, due to its accountability to the ICC Presidency and Assembly of States Parties (ASP), has strategic incentives to prioritize both impartiality and powerful state's interests when selecting situations for examination and investigation. These two priorities, however, are often contradictory: while incentives to prioritize impartiality should lead the OTP to target perpetrators of the gravest violations of human rights when states lack the willingness or ability to hold them accountable, incentives to prioritize powerful states' interests suggest that the OTP should avoid involvement in states with close ties to the permanent five members of the security council (P5).
%Thus, the OTP faces a dilemma. The functional need to prove efficacy requires the chief prosecutor to tailor her behavior to the specific interests of powerful states, but doing so may backfire, compromising the impartiality and nondiscriminatory principles of her office and the institution more broadly.

We test our expectations on impartiality and powerful states' interest using original data on ICC examinations and investigations from 2002 through 2016. Using these original data, we are able to track when the ICC moves from no involvement, to preliminary examination, to formal investigation. This type of ordinal data poses particular estimation challenges, as a state can only reach a formal proceeding once it has sequentially passed through the previous stages.\footnote{The sequential nature of ICC situation-selection is discussed further below.} Typically, scholars may employ a cumulative ordinal regression model, in which a single unobservable continuous latent variable is assumed to underlie the ordinal dependent variable. However, this type of approach would disregard that the stages of ICC involvement proceed in a sequential manner.\footnote{\citet{feinberg1980analysis} in a study of educational attainment was one of the first to note the issues of utilizing a typical ordinal framework to model a sequential process.} To deal with this, we introduce an approach that accounts for transitions between sequentially ordered stages through a latent variable representation in which the transition between any two stages is determined by a separate latent variable \citep{tutz1990sequential, albert2001sequential}.

The empirical analyses using this approach reveal several findings of note. First, neither impartiality nor powerful states' interests can fully explain the OTP's decision-making, suggesting that neither proponents nor critics of the court are fully correct. Specifically, the gravity of human rights abuses and concerns over domestic legal accountability are significant predictors of the decision to initiate a preliminary examination, but do not consistently predict the transition to formal investigation. On the other hand, powerful states' interests have a significant effect on the OTP's decision to advance to formal investigation, but are inconsistent predictors of the initial decision to initiate a preliminary examination. In other words, the OTP seems to be more driven by concerns over impartiality and legal mandate when deciding to initiate preliminary examinations, but more influenced by geopolitical concerns and powerful states interests when deciding whether to advance its preliminary examinations to the formal investigation stage.

The differential effects of these factors across the OTP's two decision stages would not have been observable using a cumulative ordinal regression model. Thus, our novel empirical approach allows us to uncover previously unacknowledged variation in the OTP's behavior, and our results demonstrate that the ICC's behavior is more nuanced than previously thought: the mandate is important but power politics is not absent from the OTP's decision calculi, and, perhaps most interestingly, different stages are driven by different considerations. Our empirical approach also allows us to assess claims that the court is biased against Africa in a more nuanced way by examining the Africa effect at each stage of the OTP's decision-making. We find that while Africa is certainly linked with a greater likelihood of ICC involvement overall, this effect is only consistent and most pronounced at the second stage, when the OTP must decide whether to initiate a formal investigation. These results are significant in that they bring rigorous empirical evidence to bear upon a key debate among scholars, policy-makers, and politicians regarding the role of the ICC in international politics and speak directly to recent and growing criticism of the court for its alleged case-selection biases.

Our findings also have implications for research on ICC effectiveness. Recent research on the court primarily focuses on understanding whether the ICC is capable of achieving international justice, promoting peace, and deterring human rights abuses \citep{appel2018shadow, hillebrecht2016deterrent, jo2016can, prorok2017compatibility}. However, if the OTP systematically selects cases in which these goals are easier (or harder) to achieve, any conclusions drawn in existing research on the effectiveness of the court may be questionable. While many existing studies do address selection concerns empirically, they use a variety of different strategies that are implemented in an ad hoc manner without a clear underlying model of OTP situation selection to guide modeling of the selection process. This may account for conflicting results in existing research on the ICC's impact, and suggests that a clear understanding of OTP situation selection is critical not just in its own right, but as an important part of understandings the ICC's broader impact. Thus, before we can fully understand the ICC's effects on international peace, justice, and deterrence, we must understand where it gets involved. By providing a more systematic understanding of OTP decision-making, therefore, this paper helps us to better conduct research on ICC effectiveness.

Finally, this paper also contributes to the broader debate about the autonomy and influence of international courts (ICs) beyond the ICC. As the international judiciary has expanded and its power increased over the past three decades, scholars have devoted significant attention to understanding whether ICs are autonomous institutions that can shape state behavior through their rulings, or whether, instead, their decisions are influenced by the states that created them. The judicial independence of ICs is also an issue of policy importance, given the increasing prominence of these institutions and the common belief that an IC's effectiveness hinges upon its independence. This paper contributes to this debate by breaking with previous research and studying the autonomy of IC \textit{case selection}, rather than IC \textit{rulings}. By studying the ICC's situation selection and advancement process, we demonstrate that case selection is a key decision point at which ICs may be influenced by geopolitical concerns. Our findings are relevant not only for understanding the behavior of ICs like the ICC, which has broad discretion to initiate preliminary and formal investigations, but for all ICs that have discretion when it comes to admissibility decisions. Our study demonstrates that the set of cases that makes it through admissibility decisions to a substantive rulings is likely non-random, and future research on IC autonomy must consider this selection process to fully understand constraints on IC autonomy.
