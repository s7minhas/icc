INTRO REVISIONS:  Crisis of legitimacy at the ICC – Africa bias, AU backing mass withdrawal; phillinpines pulling out this year, etc.  Challenges based on case selection primarily.  Other challenges claiming the court is ineffective – fails to bring ppl to justice, fails to fulfill its mandate (not sure if we should introduce this here or not).

The Rome Statute establishing the International Criminal Court (ICC) entered into force on July 1, 2002. Today, 124 states have ratified the Rome Statute, making them States Parties to the ICC. The ICC is the first permanent, international body charged with prosecuting genocide, crimes against humanity, and war crimes when domestic justice systems are either unwilling or unable to do so. Unlike previous international tribunals tasked with investigating and prosecuting individuals for war crimes, such as the International Tribunals for the Former Yugoslavia and Rwanda, the ICC’s Office of the Prosecutor (OTP) has considerable discretion when it comes to determining which cases it will pursue; the OTP’s proprio motu authority allows it to initiate investigations independent of state or UNSC referral, and even in referred cases, the court retains ultimate decision-making authority with regard to whether or not to pursue formal investigations, warrants, and trials (Danner 2003; Schabas 2011; Stahn 2009). Thus, the ICC is a novel institution, enjoying more influence and discretion in the pursuit of justice and promotion of international criminal and humanitarian law than any analogous institutions before it.

Furthermore, the ICC has become an active player in international justice. It has, to date, initiated 24 preliminary examinations, ten of which have advanced to formal investigations, and has initiated proceeding against 39 individuals as of early 2017. However, these ongoing examinations and cases represent only a small fraction of the total number which have been referred to the Court; as of September 2016, the Court had received 12,022 communications from individuals, NGOs, and IOs, pursuant to article 15 of the Rome Statute, providing information on alleged crimes falling under the Court’s jurisdiction from countries all over the world.\footnote{Office of the Prosecutor, ``Report on Preliminary Examination Activities 2016.''} 

The Court’s vast discretion in selecting cases, coupled with the large number of situations that have been referred through formal communications, begs a critical question: how does the OTP decide which situations to examine or investigate? Understanding the Court’s selection process should be of critical importance to both scholars and policy-makers, particularly in light of recent and growing criticism of the Court for its alleged case-selection biases. In particular, all of the situations in which the ICC has proceeded to the formal investigation stage involve African countries, except for one (i.e. Georgia). This so-called Africa bias has prompted criticism from prominent regional leaders such as Rwanda’s Paul Kagame and key institutions, especially the African Union, and threatens to erode the Court’s influence and legitimacy in the region (BBC News 2012; Murithi 2012).  In fact, in 2016 Burundi, South Africa, and Gambia threatened and/or initiated proceedings to leave the ICC, based upon claims of an Africa bias (Murphy 2016).

This paper systematically examines how the ICC selects its cases, and which cases it decides to escalate. Does the Court, consistent with its mandate, pursue justice in the gravest situations characterized by the worst violations of international humanitarian law, or is its decision-making influenced more by institutional constraints and resulting strategic calculations?  We develop two alternative theoretical explanations for the Court’s behavior, termed the legal/institutional incentives argument and the realist/institutional constraints argument. Each focuses on a different aspect of the incentives and constraints faced by the OTP to explain the court’s behavior.  The first builds upon the notion that the Court’s central goal is to fulfill its organizational mandate and to increase its institutional legitimacy in the service of that goal. The second argues that institutional constraints, such as the court’s lack of enforcement capabilities, generate incentives for the OTP to avoid cases that jeopardize or conflict with the interests of the most powerful states in the international system.  The institutional incentives argument leads us to hypothesize that ICC involvement is more likely in countries facing the gravest violations of human rights, and in states that lack the willingness or ability to hold individuals accountable domestically. The institutional constraints hypothesis, on the other hand, suggests that ICC involvement becomes less likely as powerful states’ interests in a given country increase.  

We focus on the ICC’s decision to both start preliminary examinations and to escalate to formal investigations, warrants, hearings, and trials, as both initiation and escalation constitute key decision points for the ICC. We thus measure ICC involvement on a scale, hypothesizing that grave human rights abuses will increase the likelihood of higher levels of ICC involvement (legal argument), while increasing ties with strong states will decrease the likelihood and level of ICC involvement. Our expectations are also directional, such that human rights abuses perpetrated by the government are expected to increase the likelihood of state-focused ICC involvement and escalation, whereas abuses perpetrated by non-state actors are anticipated to increase the risk of opposition-focused ICC involvement, as well as the escalation of such involvement.  We also anticipate that P5 ties to governments will decrease state-focused ICC involvement/escalation, while P5 ties to opposition actors will decrease the likelihood of opposition-focused ICC action.

We test these hypotheses using an original country-month dataset identifying ICC involvement from 2002 through 2015.  The empirical analyses reveal several important findings. First, while our models of ICC onset generally perform poorly, we do find that ICC involvement onset is more likely when the government and rebels intentionally target a greater number of civilians. Our models of the escalation of ICC involvement generally produce stronger results, and provide support for both the legal mandate and the institutional constraints framework.  ICC escalation is more likely when more human rights abuses have occurred and when domestic courts are weak, but is also more likely when P5 states do not have strong ties to the targeted actor. Finally, African states are not systematically more likely to experience a preliminary examination, but they are more likely to be formally investigated by the Court, suggesting the Africa bias is more complex than previously thought. 

This paper contributes to a burgeoning literature on the ICC, empirically addressing one of the key debates in contemporary scholarship and a central question among policy makers and political leaders.  Specifically, does the Court act in accordance with its mandate? Proponents of the Court view it as a transformational institution with unprecedented authority to investigate and prosecute individuals that infringe upon state sovereignty. If the Court does not fulfill its mandate by selecting the most egregious crimes against international criminal and humanitarian law, however, it cannot live up to this lofty goal. Second, this paper puts forward two competing theoretical arguments – one institutionalist and one realist – to explain ICC behavior.  It thus contributes to the vibrant scholarly debate on the role and functioning of international institutions and organizations in world politics. Finally, our results provide empirical evidence relevant to the debate over the Court’s role in Africa.  They indicate that while the court is largely unbiased in its selection of preliminary examinations, it has, thus far, favored African countries when moving to the formal investigation stage and beyond.   

This paper proceeds as follows.  We first review existing research on the ICC from both political scientists and legal scholars.  The brief discussion suggests that there is a major gap in the literature on the determinants of ICC involvement. The next section develops the paper’s two central theoretical arguments on the institutional incentives and institutional constraints of ICC involvement.  The following section presents the research design, discusses the data and measurement issues, and presents the results of our empirical analysis.  We conclude by examining the implications of these findings and providing directions for future research.
