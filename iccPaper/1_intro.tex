On September 10, 2018, US National Security Advisor John Bolton issued a virulent critique of the International Criminal Court (ICC). In his speech to the Federalist Society, Bolton called the ICC ``ineffective, unaccountable, and indeed, outright dangerous'', and referred to the court as ``fundamentally illegitimate''.\footnote{The full text of Bolton's speech is available at: https://www.aljazeera.com/news/2018/09/full-text-john-bolton-speech-federalist-society-180910172828633.html.} While these statements reflect the policy positions of a US administration that is perhaps uniquely hostile to international institutions and efforts to advance international law, the critiques raised by Bolton are not new.

In fact, they mirror similar criticisms raised by a variety of states and international actors against the ICC over the past decade. In particular, criticism of the court has centered on two central issues: first, that the court is biased against African nations and other countries in the global south. This so-called ``Africa bias'' has prompted criticism from regional leaders such as Rwanda's Paul Kagame and key institutions, especially the African Union (\emph{BBC News} 2012; Murithi 2012). In fact, in 2016 Burundi, South Africa, and the Gambia threatened and/or initiated proceedings to leave the ICC based upon claims of bias (Murphy 2016), and in 2018 the Philippines followed suit (Villamor 2018). The second central critique of the court, mirrored in Bolton's speech, is that it is ineffective, incapable of fulfilling its mandate to achieve justice and deter atrocities. This criticism has gained traction in recent years as arrest warrants against key heads of state such as Omar Al-Bashir continue to go unexecuted and the number of convictions remains at just four (Wang 2018; White 2018).

These challenges to the ICC's legitimacy and effectiveness raise fundamental questions about its ability to fulfill its core mission to pursue international peace and justice. Proponents of the Court argue that it is a transformational institution with unprecedented authority to investigate and prosecute perpetrators of atrocity crimes, while critics argue the Court is biased and ineffective. Recent criticism makes clear that we must first understand where the court gets involved to fully understand its effects. %whether it has fulfilled its mandate to hold the worst offenders accountable and to deter grave international crimes, or whether, instead, it is a biased, ineffective institution.
Yet existing literature provides little rigorous empirical evidence regarding which of these views is more accurate. We are thus unable to answer the most basic question about the functioning of the Court: how does the ICC choose where it will act?

This paper addresses this gap by examining how the ICC's Office of the Prosecutor (OTP)\footnote{As we discuss further below, the OTP has considerable discretion when it comes to determining which situations and cases it will pursue (Danner 2003; Schabas 2011; Stahn 2009). We therefore focus our theoretical argument on the OTP chief prosecutor as the key actor of interest.} selects situations for preliminary examination and formal investigation. We argue that the OTP, due to its accountability to the ICC Presidency and Assembly of States Parties (ASP), has strategic incentives to prioritize both impartiality and effectiveness when selecting situations for examination and investigation. These two priorities, however, are often contradictory: while incentives to prioritize impartiality should lead the OTP to target perpetrators of the gravest violations of human rights when states lack the willingness or ability to hold them accountable, incentives to prioritize effectiveness should lead the OTP to take powerful states' interests into account, avoiding targets who have close ties with the permanent five members of the security council (P5).\footnote{We refer to targeted actors rather than targeted countries, because the OTP can be selective in targeting either government officials or opposition groups. For example, if impartiality drives OTP behavior, the OTP should target whomever (government and/or opposition) is most responsible for grave human rights abuses.} %If the OTP is driven by effectiveness concerns, it will consider whether government or opposition groups have strong ties to P5 members before selecting targets for examination and investigation.}
Thus, the OTP faces a dilemma. The functional need to prove efficacy requires the chief prosecutor to tailor her behavior to the specific interests of powerful states, but doing so may backfire, compromising the impartiality and nondiscriminatory principles of her office and the institution more broadly.

% note from sm: you guys should check to make sure the findings are representative of the most recent set of models
We test our expectations using original data on ICC examinations and investigations from 2002 through 2016. Using these original data, we are able to track when a state moves from no investigation to preliminary to formal proceedings at the ICC. This type of ordinal data poses particular estimation challenges, as a state can only reach a formal proceeding once it has passed through the previous stages. *NEED A FOOTNOTE HERE* To deal with this, we introduce an approach that accounts for transitions between sequentially ordered stages through a latent variable representation in which the transition between any two stages is determined by a separate latent variable (Tutz 1990, Albert \& Chib 2001).\footnote{We also develop an $\sf{R}$ package to implement this approach.} The empirical analyses using this approach reveal several important findings. First, we do find that ICC involvement onset is more likely when the government and rebels intentionally target a greater number of civilians. Our models of the escalation of ICC involvement generally produce stronger results, and provide support for both the legal mandate and the institutional constraints framework. ICC escalation is more likely when more human rights abuses have occurred and when domestic courts are weak, but is also more likely when P5 states do not have strong ties to the targeted actor. Finally, African states are not systematically more likely to experience a preliminary examination, but they are more likely to be formally investigated by the Court, suggesting the Africa bias is more complex than previously thought.

Understanding the Court's selection process is of critical importance to both scholars and policy-makers, particularly in light of recent and growing criticism of the Court for its alleged case-selection biases. First, the paper addresses one of the key debates in contemporary scholarship and a central question among policy makers and political leaders that scholars have largely overlooked in research on the Court.\footnote{need to say something about extant research.} We find that the OTP decision-making is more nuanced than both its supports and critics claim. In line with its mandate, the OTP considers the gravest cases in the preliminary examination stage and complementarity is one of the most important factors across both stages, while major power politics are important in the formal examination stage. Taken together, the finding suggest that the OTP does consider its mandate especially the primacy of domestic politics, but that international politics is also an important factor in her decision-making. The results provide a middle-ground take on the OTP -- the mandate is important but power politics is not absent from her calculi.

Similarly, the findings suggest that criticisms against the ICC by the United States and other states may be overstated. In particular, the OTP's behavior is clearly not arbitrary as some claim; further, and perhaps more importantly, major powers with strong domestic justice institutions, such as the U.S. appear to have little to fear from the OTP. At the same time, while Africa is certainly linked with greater OTP involvement, it is also important to note that other factors such as civilian targeting and especially complementarity also explain ICC targeting. Thus, the findings suggest that Africa is only one of many factors that explain OTP decision-making. ???

Third, this paper helps us to better conduct research on ICC effectiveness. While scholars find mixed evidence on the impact of the ICC (XX), they largely ignore the selection issue in their work - ICC involvement -- which may explain the inconsistent findings. Our work on ICC targeting will help scholars better account for OTP decision-making in their models, consequently helping them to directly address the selection issue that biases existing research. In turn, this will allow us to better understand the ICC's impact on human rights and other related outcomes such as civil war duration.

Fourth, the paper has important implications for scholars studying international courts, tribunals, and related transitional justice institutions. As Romano (1999) writes, the contemporary period is marked by a transformational increase in international courts and tribunals to settle contentious issues and pursue international justice (XXX). Despite this, very little is known about the case selection process across these different institutions. While there is some literature on the behavior of international judges (XXX), scholars have dedicated little time studying the decision-making of prosecutors at the ICC as well as other criminal tribunals such as the international criminal tribunal for Yugoslavia or the Special Tribunal for Lebanon. As this paper argues, understanding case selection is critical to understanding the effectiveness of these institutions. In particular, scholars studying these processes need to consider both mandate-based arguments and power politics to fully understand case selection. We also see that it is necessary to theoretically and empirically to analyze the different stages of case selection, as we find different results across them. These findings likely apply to other courts and tribunals as well.

Finally, we collect original data on ICC involvement. Scholars can use the new data to test our novel theories on the ICC such as ...


%  kazembe
% In this paper, we propose working with ordinal representation of the length of the waiting interval as an alternative approach to modelling waiting time data. The ordinal responses arise by categorizing the continuous outcomes (i.e., the interval in months) by adjacent intervals along the continuous scale. The observed response can be regarded as the result of a sequential process in which each time point can be reached successively. 

% The sequential ordinal model, as described by Albert and Chib (1997, 2001), can be used to analyze such categorical responses that occur in sequential order. The sequential ordinal model, also referred to as the continuation ratio model, is equivalent to the most commonly used cumulative ordinal model where the distribution function is the extreme value distribution (Laara and Matthews 1985; Albert and Chib 2001). For various extensions and comparisons among these models, see the overview by Liu and Agresti (2005). Tutz (2003) showed that the sequential ordinal model belongs to the multivariate exponential family, and the generalized linear model framework applies. 

% Applications of the sequential ordinal model in the analysis of event history demographic data, to our knowledge, are few. Such a use, however, is common in several other fields. In epidemiological studies, for instance Knorr-Held et al (2002) applied both cumulative and sequential ordinal models to map disease-specific cancer incidence data. 